\chapter{Einleitung}
Das Thema der Projektarbeit ist die umsetzung einer modernen \ac{CMS} Lösung. Hierbei wurde besonders auf eine von grund auf modulare Architektur geachtet. Diese ist umgesetzt mithilfe des \ac{MVVM} Entwurfsmusters. Ziel dabei ist es, eine stabile und leicht zu erweiternde Basis für eine Vielzahl von Implementationen für Webanwendungen zu schaffen.

\section{Motivation}
% TODO: Maybe add some studies here?
Mehr denn je ist es in der Softwareentwicklung erforderlich, schnell und zuverlässig auf Kundenwünsche reagieren zu können. \acs{CMS} Lösungen schaffen eine solide Grundlage, die erprobte Lösungen für immer wiederkehrende Proleme, wie beispielsweise Authentifikation und Gruppenmanagement bereitstellen.

\section{Aufbau der Arbeit}
\begin{description}
\item[Kapitel 1:]{Das erste (dieses) Kapitel behandelt das Thema und die Motivation der Projektarbeit. Desweiteren wird der Aufbau der Arbeit erläutert}
\item[Kapitel 2:]{In der theoretischen Betrachtung werden grundlegende Konzepte eingeführt. Der Leser entwickelt Hintergrundwissen, welches insbesonders in der Umsetzung benötigt werden.}
\item[Kapitel 3:]{Unter dem Sitchwort Methodik werden Technologieentscheidungen erörtert, die im Laufe des Projektes getroffen wurden. Für verschiedene Probleme werden Lösungen diskutiert und eine Entscheidung begründet.}
\item[Kapitel 4:]{Die Umsetzung behandelt die tatsächliche Implementation des Projektes. Wichtige Datenstrukturen und Spezifikationen werden erläutert und diverse Code-Listings geben einen umfassenden Überblick über die Realisierung.}
\item[Kapitel 5:]{Im fünften Kapitel werden anfängliche Entscheidungen und Betrachtungen retrospektivischaufgegriffen und in Hinsicht auf die Erfahrungen im Projekt diskutiert}
\item[Kapitel 6:]{Am Ende der Arbeit steht eine Zusammenfassung. Es wird auf die tatsächlich Umsetzung der Ziele eingegangen und ein Ausblick gegeben.}
\end{description}
