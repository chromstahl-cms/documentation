\chapter{Methodik}
\section{Vergleich von Web-Frontend Sprachen}
Es gibt eine Vielzahl von möglichen Programmiersprachen zur Erstellung von
Web-Frontends. Zu Beginn des Projektes wurden daher verschiedene Sprachen
evaluiert. Nachfolgend die Ergebnisse unserer Betrachtungen.
\subsection{Javascript}
Nach dem ersten Erscheinen am 4. Dezember
1995\footnote{https://web.archive.org/web/20070916144913/http://wp.netscape.com/newsref/pr/newsrelease67.html}
entwickelte sich \ac{JS} neben \ac{HTML} und \ac{CSS} schnell zu einer der Kern
Technologien des Web.\\
Die Syntax von \ac{JS} basiert, genau wie etwa C oder
Java, auf geschweiften Klammern. Es werden diverse Paradigma wie Funktionale und
Objekt Orientierte Programmierung unterstützt. \acl{JS} liefert eine umfangreiche
Standardbibliothek mit einer einer Vielzahl von
\ac{API}s zum Manipulieren eines \ac{DOM}s. Die Sprache ist
dynamisch\footnote{https://en.wikipedia.org/wiki/Type\_system\#Combining\_static\_and\_dynamic\_type\_checking}
typisiert.
\subsection{\acl{TS}}
\ac{TS} ist ein von Microsoft entwickeltes Superset von
\acl{JS}\footnote{https://www.typescriptlang.org}. Es erweitert \ac{JS} um eine
optional verwendbare strikte Typisierung, die beim Übersetzen zu \ac{JS}
überprüft wird. Dies hat zwei Vorteile gegenüber einer dynamischen Sprache wie
beispielsweise \acl{JS}\footnote{https://pchiusano.github.io/2016-09-15/static-vs-dynamic.html}:
\begin{description}
  \item[Entdecken von Fehlern zur Kompilierzeit]{Viele semantische Fehler werden
    während des Kompilierens erkannt. Hierzu gehören Fehler wie das Übergeben
    von Parametern eines falschen Typs in eine Funktion oder das aufrufen einer
    Funktion, die nicht existiert}
  \item[Bessere Lesbarkeit]{Statisch typisierter Quellcode ist für den
      Programmierer einfacher lesbar, da Typen von Parametern und Variablen
      direkt ersichtlich sind und nichts aus dem Name oder dem Kontext
      herausgefunden werden müssen}
  \item[Bessere Editor Untersützung]{Dadurch, dass Typinformationen zur
      Kompilierzeit zur Verfügung stehen, kann der Editor dem Programmierer eine
      bessere Autovervollständigung bieten}
\end{description}
Ein Nachteil von \acl{TS} ist die Tatsache, dass Typinformationen von externen
Softwarebibliotheken für die vollständige Überprüfung des Typsystems
vorhanden sein muss. Ein Großteil von \ac{JS} Bibliotheken wird ohne diese
Informationen geliefert, weshalb für diese Typinformationen von der Community
benutzt werden müssen.\\
Da die Vorteile von statischer Typisierung den Nachteilen überwiegen, wird
\acl{TS} als Frontend Sprache in Chromstahl verwendet.
\section{Vergleich von Frontend-Frameworks}
\subsection{Vue}
\subsection{JQuery}
\subsection{Eigene Implementation}
\section{Vergleich von Web-Backend Sprachen}
\subsection{Java}
\subsection{Go}
\subsection{Javascript}
\section{Vergleich von Backend-Frameworks}
\subsection{Spring}
\subsection{Play}
\subsection{JavaEE}
\section{Vergleich von Datenbank-Lösungen}
\subsection{Relational Database}
\subsection{Document Oriented Database}