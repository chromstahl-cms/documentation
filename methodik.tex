\chapter{Methodik}
\section{Vergleich von Web-Frontend Sprachen}
Es gibt eine Vielzahl von möglichen Programmiersprachen zur Erstellung von
Web-Frontends. Zu Beginn des Projektes wurden daher verschiedene Sprachen
evaluiert. Nachfolgend die Ergbnisse unserer Betrachtungen.
\subsection{Javascript}
Nach dem ersten Erscheinen am 4. Dezember
1995\footnote{https://web.archive.org/web/20070916144913/http://wp.netscape.com/newsref/pr/newsrelease67.html}
entwickelte sich \ac{JS} neben \ac{HTML} und \ac{CSS} schnell zu einer der Kern
Technologien des Web.\\
Die Syntax von \ac{JS} basiert, genau wie etwa C oder
Java, auf geschweiften Klammern. Es werden diverse Paradigma wie Funktionale und
Objekt Orientierte Programmierung untersützt. \acl{JS} liefert eine umfangreiche
Standardbibliothek mit einer einer Vielzahl von
\ac{API}s zum Manipulieren eines \ac{DOM}s. Die Sprache ist dynamisch\footnote{https://en.wikipedia.org/wiki/Type\_system\#Combining\_static\_and\_dynamic\_type\_checking} und
``duck''\footnote{https://en.wikipedia.org/wiki/Duck\_typing} typisiert. % TODO % better wording
\subsection{Typescript}
\subsection{Web-ASM}
\section{Vergleich von Frontend-Frameworks}
\subsection{Vue}
\subsection{JQuery}
\subsection{Eigene Implementation}
\section{Vergleich von Web-Backend Sprachen}
\subsection{Java}
\subsection{Go}
\subsection{Javascript}
\section{Vergleich von Backend-Frameworks}
\subsection{Spring}
\subsection{Play}
\subsection{JavaEE}
\section{Vergleich von Datenbank-Lösungen}
\subsection{Relational Database}
\subsection{Document Oriented Database}